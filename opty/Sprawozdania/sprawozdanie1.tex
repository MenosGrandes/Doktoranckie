%%%%%%%%%%%%%%%%%%%%%%%%%%%%%%%%%%%%%%%%%
% University Assignment Title Page 
% LaTeX Template
% Version 1.0 (27/12/12)
%
% This template has been downloaded from:
% http://www.LaTeXTemplates.com
%
% Original author:
% WikiBooks (http://en.wikibooks.org/wiki/LaTeX/Title_Creation)
%
% License:
% CC BY-NC-SA 3.0 (http://creativecommons.org/licenses/by-nc-sa/3.0/)
% 
% Instructions for using this template:
% This title page is capable of being compiled as is. This is not useful for 
% including it in another document. To do this, you have two options: 
%
% 1) Copy/paste everything between \begin{document} and \end{document} 
% starting at \begin{titlepage} and paste this into another LaTeX file where you 
% want your title page.
% OR
% 2) Remove everything outside the \begin{titlepage} and \end{titlepage} and 
% move this file to the same directory as the LaTeX file you wish to add it to. 
% Then add \input{./title_page_1.tex} to your LaTeX file where you want your
% title page.
%
%%%%%%%%%%%%%%%%%%%%%%%%%%%%%%%%%%%%%%%%%
%\title{Title page with logo}
%----------------------------------------------------------------------------------------
%	PACKAGES AND OTHER DOCUMENT CONFIGURATIONS
%----------------------------------------------------------------------------------------

\documentclass[12pt]{article}
\usepackage[english]{babel}
\usepackage{polski}
\usepackage[utf8x]{inputenc}
\usepackage{amsmath}
\usepackage{graphicx}
\usepackage[colorinlistoftodos]{todonotes}
\usepackage[calc]{datetime2}
\usepackage{listings}
\usepackage{siunitx}
\usepackage{placeins}
\usepackage{listings}
\usepackage{tikz} % To generate the plot from csv
\usepackage{pgfplots}
\pgfplotsset{
    legend entry/.initial=,
    every axis plot post/.code={%
        \pgfkeysgetvalue{/pgfplots/legend entry}\tempValue
        \ifx\tempValue\empty
            \pgfkeysalso{/pgfplots/forget plot}%
        \else
            \expandafter\addlegendentry\expandafter{\tempValue}%
        \fi
    },
}


\pgfplotsset{compat=newest} % Allows to place the legend below plot
\usepgfplotslibrary{units} % Allows to enter the units nicely

\sisetup{
  round-mode          = places,
  round-precision     = 2,
}

\begin{document}

\begin{titlepage}

\newcommand{\HRule}{\rule{\linewidth}{0.5mm}} % Defines a new command for the horizontal lines, change thickness here

\center % Center everything on the page
 
%----------------------------------------------------------------------------------------
%	HEADING SECTIONS
%----------------------------------------------------------------------------------------

\textsc{\LARGE Politechnika Łódzka}\\[1.5cm] % Name of your university/college
\textsc{\Large Szybkie algorytmy}\\[0.5cm] % Major heading such as course name

%----------------------------------------------------------------------------------------
%	TITLE SECTION
%----------------------------------------------------------------------------------------

\HRule \\[0.4cm]
{ \huge \bfseries Techniki zwiększenie efektywności algorytmów }\\[0.4cm] % Title of your document
\HRule \\[1.5cm]
 
%----------------------------------------------------------------------------------------
%	AUTHOR SECTION
%----------------------------------------------------------------------------------------



\begin{flushleft}\large

\begin{center} \emph{Prowadzący zajęcia:} \end{center}
\begin{center}
prof. dr hab. Mykhaylo \textsc{Yatsymirskyy}
\end{center} % Supervisor's Name
\end{flushleft}


% If you don't want a supervisor, uncomment the two lines below and remove the section above
\Large \emph{Autor:}\\
Filip \textsc{Rynkiewicz}\\[2cm] % Your name


%----------------------------------------------------------------------------------------
%	LOGO SECTION
%----------------------------------------------------------------------------------------

\includegraphics[scale=0.36]{logo.png}\\[1.3cm]
 

%----------------------------------------------------------------------------------------
%	DATE SECTION
%----------------------------------------------------------------------------------------

{\large \today}\\[5cm] % Date, change the \today to a set date if you want to be precise

\vfill % Fill the rest of the page with whitespace

\end{titlepage}





\lstset{language=C++,
                 basicstyle=\ttfamily\footnotesize,
                keywordstyle=\color{blue}\ttfamily,
                stringstyle=\color{red}\ttfamily,
                commentstyle=\color{green}\ttfamily,
                morecomment=[l][\color{magenta}]{\#},
breaklines=true
}


\pgfplotsset{
          width=\linewidth, % Scale the plot to \linewidth
          grid=major, 
          grid style={dashed,gray!30},
          xlabel=Ilosc elementow w tablicy, % Set the labels
          ylabel=Cos pomieszane SPRAWDZ ,
          x tick label style={rotate=0,anchor=east,align=right,yshift=-1.5ex},
          scaled y ticks = false,  
          scaled x ticks = false,   
		  legend pos=north west,
 		  samples=61,
    	  width=6in,
    	  height=4in,
          ymin=0,
          try min ticks =6,
          max space between ticks=15000pt,
          x tick label style=
          {
          /pgf/number format/fixed,
     	 /pgf/number format/1000 sep = \thinspace % Optional if you want to replace comma as the 1000 separator 
      	  },
      	  y tick label style=
          {
          /pgf/number format/fixed,
     	 /pgf/number format/1000 sep = \thinspace % Optional if you want to replace comma as the 1000 separator 
      	  },
      	  xmin=0
}
\section{Algorytm sortowania przez wstawianie}

\subsection*{Dla dwóch elementów}

Początkowym krokiem algorytmu jest posortowanie  par $(x,y) $ w zbiorze $V  $, gdzie każda para $(x,y) \in V$,  tak aby pierwsza liczba $x$  była zawsze liczba mniejsza od liczby $y$. Indeksy liczby $x$ jest zawsze o jeden mniejszy od indeksu liczby $y$ w zbiorze.\\
 Pierwszym krokiem tego sortowania będzie wybranie  pary $(x',y')$. Pary wybierane sa poprzez przesuwanie od indeksu 2 zbioru $V$, poniewaz zakladamy ze pierwsza para jest posortowana, zawsze o 2 indeksy. Kazde przejscie zaczyna sie od elementu $V[i+2]$\footnote{Indeks $i+2$ ze zbioru $V$, gdzie i jest kolejna iteracja algorytmu.}.\\
Po wybraniu pary $(x',y')$ nastepuje porownywanie elementu $y'$ z elementem $z$, ktory poprzedza wybrana pare oraz indeks $z \in \lvert V \rvert $. Dopoki  $z > y'$  wykonuje sie przesuniecie calej pary przed liczbe $z$. Za kazdym razem liczba $z$ jest liczba poprzedzajaca liczbe $x'$.
Jezeli  $z < y'$ algorytm przechodzi do porownania $z > x'$ . Jezeli zostanie spelniony ten warunek liczba mniejsza z pary zostaje przestawiona przed liczba $z$. 
\subsubsection*{Wyniki}
\newpage
\subsubsection*{Kod}
\begin{lstlisting}
void sort(std::vector<int> &toSort)
{
 const int sizeOfArray=toSort.size()-(toSort.size()%2);
 for(int i=0; i<sizeOfArray; i+=2)
     {
      if(toSort[i] > toSort[i+1])
         {
          std::swap(toSort[i],toSort[i+1]);
         }
     }
 for(int i=2; i<sizeOfArray; i+=2)
    {
      const int pom1 = toSort[i];
      const int pom2 = toSort[i+1];
      int j = i-1;
	  while(j>=0 && toSort[j]>pom2)
	   {
	    toSort[j+2] =  toSort[j];
	    j--;
	   }
	  toSort[j+2] = pom2;
	  toSort[j+1] = pom1;
	  while(j>=0 && toSort[j]>pom1)
	   {
	    toSort[j+1] = toSort[j];
	    --j;
	   }
	  toSort[j+1] = pom1;
	}
	if(toSort.size()%2==1)
	   {
	    const int pom = toSort[toSort.size()-1];
	    int k = toSort.size()-2;
	    while(k>=0 && toSort[k]>pom)
	     {
	      toSort[k+1] = toSort[k];
	      --k;
		 }
	   toSort[k+1] = pom;
	    }
}
\end{lstlisting}
\begin{figure}[h!]
  \begin{center}
    \begin{tikzpicture}
      \begin{axis}
      
        \addplot[green,mark=x,legend entry=$Normalny$] 
        table[x=column 1,y=column 2,col sep=comma] {sortInsertTupleTest.csv}; 
    
        \addplot[red,mark=x,legend entry=$Dla\;par$] 
        table[x=column 1,y=column 3,col sep=comma] {sortInsertTupleTest.csv};
       
      \end{axis}
    \end{tikzpicture}
    \caption{Wykres dla losowych elementow tablicy}
  \end{center}
\end{figure}
\FloatBarrier

\section{Algorytm sortowania bąbelkowego}
\subsection{Dla dwóch elementów}
Podstawowa wersja tego algorytmu polega na porownywaniu ze soba dwoch kolejnych elementow $(x,y) \in V$ i zmianie ich kolejnosci, majac tylko jeden babelek ktory wyplywa na poczatek lub na koniec zbioru. \\
Zakladajac ze mamy porownywac dwie liczby, zostalo przyjete ze sa dwa babelki. Jeden ktory idzie na poczatek zbioru  $V$ oraz drugi ktory idzie na koniec zbioru $V$. \\
Dla kazdej pary $(x',y') \in V$ skladajacej sie z kolejnych elementow ze zbioru $V$:
\begin{itemize} 
\item Posortuj pare $(x',y)'$ rosnaco
\item Dla kazdej liczby $z$ poprzedzajacej $x'$ zamien ze soba te elementy jezeli\begin{equation*}x' < z\end{equation*} 
\item  Dla kazdej liczby $w$ nastepujacej $y'$ zamien ze soba te elementy jezeli \begin{equation*}y'>w\end{equation*}
 \end{itemize}  
\subsubsection*{Kod}

\begin{lstlisting}
void sort(std::vector<int> &toSort)
{
 for(int i= 0; i<(toSort.size()-1); i++)
 {
  int minElem=i,maxElem=i+1;
  if(toSort[minElem]>toSort[maxElem])
   {
    std::swap(toSort[minElem],toSort[maxElem]);
   }
  int j=i;
  while(j>0 && toSort[minElem]<toSort[minElem-1])
   {
    std::swap(toSort[minElem],toSort[minElem-1]);
    --j;
    minElem--;
   }
  int j2=i+1;
  while(j2<(toSort.size()-1) && toSort[maxElem]>toSort[maxElem+1])
   {
    std::swap(toSort[maxElem],toSort[maxElem+1]);
    ++j2;
    maxElem++;
   } 
 }
}
\end{lstlisting}
\subsubsection*{Wyniki}
\begin{figure}[h!]
  \begin{center}
    \begin{tikzpicture}
      \begin{axis}
      
        \addplot[green,mark=x,legend entry=$Normalny$] 
        table[x=column 1,y=column 2,col sep=comma] {sortBubbleTupleTest.csv}; 
    
        \addplot[red,mark=x,legend entry=$Dla\;par$] 
        table[x=column 1,y=column 3,col sep=comma] {sortBubbleTupleTest.csv};
       
      \end{axis}
    \end{tikzpicture}
    \caption{Wykres dla losowych elementow tablicy}
  \end{center}
\end{figure}
\FloatBarrier

\section{Algorytm sortowania przez wybieranie}
\subsection{Dla dwóch elementów}
\subsubsection*{Kod}
\begin{lstlisting}
void sort(std::vector<int> &toSort)
{
 int vectorSize=0;
 if(toSort.size()%2!=0)
  {
   vectorSize++;
   std::iter_swap((std::min_element(toSort.begin(),toSort.end())),toSort.begin());
  }
 std::vector<int>::iterator _begin = toSort.begin()+vectorSize;
 std::vector<int>::iterator _end = toSort.end() - 1;
 while (_begin < _end)
  {
   std::vector<int>::iterator it=_begin,_min=it,_max=it;
   for (it = _begin; it <= _end; ++it)
    {
     if ((*it) < (*_min))
      {
	   _min = it;
	  }
	 else if ((*it) > (*_max))
	  {
	   _max = it;
	  }
    }
 std::iter_swap(_min,_begin);
 if(_begin==_max)
  {
   _max=_min;
  }
 std::iter_swap(_max,_end);
 ++_begin;
 --_end;
  }
}
\end{lstlisting}
\subsubsection*{Wyniki}
\begin{figure}[h!]
  \begin{center}
    \begin{tikzpicture}
      \begin{axis}
      
        \addplot[green,mark=x,legend entry=$Normalny$] 
        table[x=column 1,y=column 2,col sep=comma] {sortSelectionTupleTest.csv}; 
    
        \addplot[red,mark=x,legend entry=$Dla\;par$] 
        table[x=column 1,y=column 3,col sep=comma] {sortSelectionTupleTest.csv};
       
      \end{axis}
    \end{tikzpicture}
    \caption{Wykres dla losowych elementow tablicy}
  \end{center}
  \end{figure}
\FloatBarrier


\end{document}
