

\section{BitonicSort}
Jest to algorytm sortowania równoległego, która wykorzystuje sekwencje
bitoniczne.\\
Sekwencja bitoniczna można nazwać taka sekwencje która najpierw jest
rosnąca potem malejąca. Zbiór $V [0 \ldots n−1]$ można nazwać sekwencja biton-
iczna jeżeli istnieje taki indeks $i$ gdzie $0 ‹ i ‹ n − 1$ :
\begin{equation*}
x_0 \leq x_1 \ldots \leq x_i \; oraz \; x_i \geq x_{i+1} \ldots \geq x_{n−1}
\end{equation*}
Aby posortować kolekcje danych o długości n z dwóch sekwencji o długości$\frac{n}{2}$
potrzeba $\ln(n)$ porównań.
\begin{equation*}
T(n) = ln(n) + T( \frac{n}{2})
\end{equation*}
Ponieważ jest to równanie rekurencyjne trzeba rozwinąć je do postaci nierekuren-
cyjnej.
\begin{align*}
T(n) &= \ln(n) + \ln(n) − 1 + \ln(n) − 2 + \ldots + 1 \\
&= \ln(n) \cdot \frac{\ln+1}{2} \\
&= O(n \cdot \ln(n)^2 )
\end{align*}

