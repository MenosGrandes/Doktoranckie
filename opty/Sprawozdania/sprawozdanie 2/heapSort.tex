\section{HeapSort}
Sortowanie przez kopcowanie polega na wykorzystaniu kopca.\\
 Podstawowymi akcjami wykonywanymi przez algorytm jest:
\begin{itemize}
\item Tworzenie kopca,
\item sortowanie.
\end{itemize}

\subsubsection{Kopiec}
Jest to kontener danych podobny do drzewa binarnego\cite{heapsort}. Każdy element w kopcu posiada 3 inne elementu : swojego przodka oraz dwójkę dzieci, element \textit{root} nie posiada przodka ponieważ jest pierwszym elementem. 
Element \textit{przodek} zawsze posiada wartość większa od wartości swoich dzieci.	\\
Ogolny wzor prowadzacy do obliczenia zlozonosci czasowej tego algorytmu to:
\begin{equation}
T(N) = T_{buildheap}(N) +\sum_{k=1}^{n-1} T_{heapify}k + O(n-1)
\end{equation}

Aby zawsze na pierwszym miejscu znajdował się maksymalny element zbioru kopiec posiada metodę o nazwie $Heapify$. Jest ona odpowiedzialna za stworzenie takiego kopca którego każdy element jest uporządkowany tak aby wartość rodzica była wartością większa niż wartości jego dzieci.\\
Złożoność obliczeniowa takiej funkcji to :
\begin{align*}
T(N) &= T(\frac{2n}{3}) + O(1) \\
T(N) &=O(\log n)
\end{align*} 
\par Kolejna funkcja wykorzystywana w tym sortowaniu jest tworzenie kopca $BuildHeap$.
J
\subsection{Klasyfikacja}
\subsubsection{Optymistyczna}

\subsubsection{Pesymistyczna}


\subsubsection{Średnia}
