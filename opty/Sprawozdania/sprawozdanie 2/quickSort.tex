

\section{Quicksort}
Sortowanie to polega na wybraniu jednego elementu $p$, który może być środkowym, skrajnie lewym, skrajnie prawym lub losowym elementem zbioru. W następnym kroku elementy nie większe niż $p$ zostają ustawione na lewo tej wartości, a nie mniejsze na prawo. W ten sposób powstaną nam dwie części tablicy (niekoniecznie równe), gdzie w pierwszej części znajdują się elementy nie większe od drugiej. Następnie każdą z tych pod tablic sortujemy osobno według tego samego schematu. 
\subsection{Klasyfikacja}
\subsubsection{Optymistyczna}
Jeżeli funkcja dzieląca zbiory o rozmiarze $n$ będzie dzieliła je zawsze na prawie równe części o rozmiarze $\frac{n}{2}$, można ustalić ze jest to najlepszy przypadek dla tego sortowania. Dzielenie takie następuje gdy wartość $p$ będzie mediana zbioru \cite{quicksort}
\begin{equation*}
T(n) = 2T(\frac{n}{2}) +\alpha n
\end{equation*}
Zakładając ze powyższa funkcja jest funkcja rekurencyjna następują zmiany dla kolejnej iteracji
\begin{align*}
 T(n) &= 2(2T(\frac{n}{4}) + \alpha \frac{n}{2}) + \alpha n \\
      &= 2^{2}T( \frac{n}{4}) + 2 \alpha n \\
      &= 2^{3}T( \frac{n}{8}) + 3 \alpha n
 \end{align*}
Rekurencyjne iteracje będą wykonywane do momentu $n = 2^{k}$, gdzie $\alpha$ jest  stalą, $k$ ilością iteracji, a $n$ rozmiarem zbioru początkowego. Czyli przyjmując ze $k = \log n$ :
\begin{equation*}
T(n) =nT(1) +\alpha n \log n
\end{equation*}
W notacji duze O bedzie to zatem:
\begin{equation*}
O(n \log n)
\end{equation*}
\subsubsection{Pesymistyczna}
Najgorszy przypadek w tym algorytmie występuje w sytuacji kiedy algorytm będzie dzielił tablice na pod tablice w sytuacji kiedy jedna z nich będzie zawierała 1 element.
\begin{equation*}
T(n) =T(1) +T(n−1) + \alpha n
\end{equation*}
Dla każdej iteracji rekurencji wyrażenie to będzie przedstawione jako 
\begin{align*}
T(n) &=T(n−1) +T(1) + \alpha n \\
	 &= T(n−2) +T(1) +\alpha(n−1)] +T(1) + \alpha n \\
	 &= T(n−3) + 3T(1) +\alpha(n−2 +n−1 +n) \\
	 &= T(n−i) +iT(1) +\alpha(n−i+ 1 + \cdot + n−2 +n−1 +n) \\
	 &= T(n−i) +iT(1) +\alpha \sum_{j=0}^{n-2}(n-j) \\ 
	 &= nT(1) +\alpha(n(n−2)−\frac{(n−2)(n−1)}{2})
 \end{align*}
Rekurencja wykonuje się dopóki $i=n-1$, ponieważ nie możne być sytuacji gdzie $n-i <0$.
Zlozonosc obliczeniowa wychodzi wiec :
\begin{equation*}
O(n^2)
\end{equation*}
\subsubsection{Średnia}
Zakładając że piwot $p$ będzie wybierany tak aby partycjonowanie zbioru $V $ dzieliło go na zbiory $V_1$ i $V_2$ to zawsze prawdą będzie że długość tych podzbiorów będzie zawierała się w zbiorze od $0$ do $ \norm{V}-1$ \cite{quicksortAVG}.

Aby posortować \textit{n} elementów potrzeba \textit{T} czasu :
\begin{equation*}
T(n)=T(i)+T(n-i)+c \cdot n
\end{equation*}
Ponieważ $i$ jest wartośćią z przedziału $0$ do $n-1$, średnia wartość $T$ to :
\begin{equation*}
T(i)=\frac{1}{n}\sum_{j=0}^{n-1}T(j)
\end{equation*}
Skoro $n-i$ może mieć taką samą wartosć jak $i$:
\begin{equation*}
T(n-1) = T(i)
\end{equation*}
Zatem :
\begin{equation*}
T(n)=\frac{2}{n} \Bigg( \sum_{j=0}^{n-1}T(j)\Bigg) + c \cdot n
\end{equation*}
Usuwając $n$ z ułamka  :
\begin{equation*}
n \cdot T(n)=2 \cdot \Bigg( \sum_{j=0}^{n-1}T(j)\Bigg) + c \cdot n^{2}
\end{equation*}
Ponieważ równanie musi być prawdziwe dla każdej liczby trzeba zamienić $n$ na przykład na $n-1$ :
\begin{equation*}
(n-1) \cdot T(n-1)=2 \cdot \Bigg( \sum_{j=0}^{n-2}T(j)\Bigg) + c \cdot (n-1)^{2}
\end{equation*}
Wiec prawdą będzie :
\begin{align*}
n \cdot T(n)&=2 \cdot \Bigg( \sum_{j=0}^{n-1}T(j)\Bigg) + c \cdot n^{2}\\
(n-1) \cdot T(n-1)&=2 \cdot \Bigg( \sum_{j=0}^{n-2}T(j)\Bigg) + c \cdot (n-1)^{2}
 \end{align*}
 Odejmując drugie równanie od pierwszego, usuwając $c$ i grupując równanie otrzymamy :
 \begin{equation*}
 nT(n)=(n+1)T(n-1)+2cn
 \end{equation*}
Rozwiązując równanie rekurencyjne i grupując je otrzymamy :
\begin{equation*}
 \frac{T(n)}{(n+1)} = \frac{T(1)}{2} + 2 c \sum_{j=3}^{n+1} \frac{1}{j}
\end{equation*}
Ponieważ $n$ staję się bardzo duże, $\sum_{j=3}^{n+1} \frac{1}{j}$ dochodzi do $\ln(n) +\gamma $, gdzie $\gamma$ jest stałą Eulera.\\
Zatem :
\begin{equation*}
 \frac{T(n)}{n+1} = \frac{T(1)}{2} + 2 c \cdot \ln(n) + 2 c \gamma = \ln(n) + c_2 = O \ln(n) 
\end{equation*}
Złozoność wynosi zatem:
\begin{equation*}
 T(n)=O (n \cdot \ln(n)) 
 \end{equation*}

