%%%%%%%%%%%%%%%%%%%%%%%%%%%%%%%%%%%%%%%%%
% University Assignment Title Page 
% LaTeX Template
% Version 1.0 (27/12/12)
%
% This template has been downloaded from:
% http://www.LaTeXTemplates.com
%
% Original author:
% WikiBooks (http://en.wikibooks.org/wiki/LaTeX/Title_Creation)
%
% License:
% CC BY-NC-SA 3.0 (http://creativecommons.org/licenses/by-nc-sa/3.0/)
% 
% Instructions for using this template:
% This title page is capable of being compiled as is. This is not useful for 
% including it in another document. To do this, you have two options: 
%
% 1) Copy/paste everything between \begin{document} and \end{document} 
% starting at \begin{titlepage} and paste this into another LaTeX file where you 
% want your title page.
% OR
% 2) Remove everything outside the \begin{titlepage} and \end{titlepage} and 
% move this file to the same directory as the LaTeX file you wish to add it to. 
% Then add \input{./title_page_1.tex} to your LaTeX file where you want your
% title page.
%
%%%%%%%%%%%%%%%%%%%%%%%%%%%%%%%%%%%%%%%%%
%\title{Title page with logo}
%----------------------------------------------------------------------------------------
%	PACKAGES AND OTHER DOCUMENT CONFIGURATIONS
%----------------------------------------------------------------------------------------

\documentclass[12pt]{article}
\usepackage[english]{babel}
\usepackage{polski}
\usepackage[utf8x]{inputenc}
\usepackage{amsmath}
\usepackage{graphicx}
\usepackage[colorinlistoftodos]{todonotes}
\usepackage[calc]{datetime2}
\usepackage{listings}
\usepackage{siunitx}
\usepackage{placeins}
\usepackage[figurename=Rys.]{caption}
\usepackage{subcaption}
\usepackage{listings}
\usepackage{tikz} % To generate the plot from csv
\usepackage{mathtools}
\usepackage{titlesec}
\usepackage{hyperref}
 \usepackage{url}
\usepackage{pgfplots}
\pgfplotsset{
    legend entry/.initial=,
    every axis plot post/.code={%
        \pgfkeysgetvalue{/pgfplots/legend entry}\tempValue
        \ifx\tempValue\empty
            \pgfkeysalso{/pgfplots/forget plot}%
        \else
            \expandafter\addlegendentry\expandafter{\tempValue}%
        \fi
    },
}
\captionsetup[subfigure]{skip=30pt} % global setting for subfigure
\newcommand{\lstlistingnames}{Algorytmie}% Listing -> Algorithm
\renewcommand{\lstlistingname}{Algorytm}% Listing -> Algorithm
\renewcommand{\lstlistlistingname}{List of \lstlistingname s}
\renewcommand{\lstlistingname}{Algorytm}% Listing -> Algorithm
\newcommand{\wyjT}{gdzie $T$ to oszacowana ilość operacji a $n$ rozmiar tablicy do posortowania.\\}



\pgfplotsset{compat=newest} % Allows to place the legend below plot
\usepgfplotslibrary{units} % Allows to enter the units nicely

\sisetup{
  round-mode          = places,
  round-precision     = 2,
}

\setcounter{secnumdepth}{4}
\setcounter{tocdepth}{4}
\addto\captionsenglish{% Replace "english" with the language you use
  \renewcommand{\contentsname}%
    {Spis treści}%
}


\begin{document}

\begin{titlepage}

\newcommand{\HRule}{\rule{\linewidth}{0.5mm}} % Defines a new command for the horizontal lines, change thickness here

\center % Center everything on the page
 
%----------------------------------------------------------------------------------------
%	HEADING SECTIONS
%----------------------------------------------------------------------------------------

\textsc{\LARGE Politechnika Łódzka}\\[1.5cm] % Name of your university/college
\textsc{\Large Szybkie algorytmy}\\[0.5cm] % Major heading such as course name

%----------------------------------------------------------------------------------------
%	TITLE SECTION
%----------------------------------------------------------------------------------------

\HRule \\[0.4cm]
{ \huge \bfseries Badanie efektywności szybkich algorytmów sortowania  }\\[0.4cm] % Title of your document
\HRule \\[1.5cm]
 
%----------------------------------------------------------------------------------------
%	AUTHOR SECTION
%----------------------------------------------------------------------------------------



\begin{flushleft}\large

\begin{center} \emph{Prowadzący zajęcia:} \end{center}
\begin{center}
prof. dr hab. Mykhaylo \textsc{Yatsymirskyy}
\end{center} % Supervisor's Name
\end{flushleft}


% If you don't want a supervisor, uncomment the two lines below and remove the section above
\Large \emph{Autor:}\\
Filip \textsc{Rynkiewicz}\\[2cm] % Your name


%----------------------------------------------------------------------------------------
%	LOGO SECTION
%----------------------------------------------------------------------------------------

\includegraphics[scale=0.36]{logo.png}\\[1.3cm]
 

%----------------------------------------------------------------------------------------
%	DATE SECTION
%----------------------------------------------------------------------------------------

{\large \today}\\[5cm] % Date, change the \today to a set date if you want to be precise

\vfill % Fill the rest of the page with whitespace

\end{titlepage}





\lstset{language=C++,
                 basicstyle=\ttfamily\footnotesize,
                keywordstyle=\color{blue}\ttfamily,
                stringstyle=\color{red}\ttfamily,
                commentstyle=\color{green}\ttfamily,
                morecomment=[l][\color{magenta}]{\#},
				breaklines=true,
    			numbers=left,
    			    postbreak=\raisebox{0ex}[0ex][0ex]{\ensuremath{\color{red}\hookrightarrow\space}}
}


\pgfplotsset{
          width=\linewidth, % Scale the plot to \linewidth
          grid=major, 
          grid style={dashed,gray!30},
          xlabel=Ilość elementów w tablicy, % Set the labels
          ylabel=Czas wykonania [ms] ,
          x tick label style={rotate=0,anchor=east,align=right,yshift=-1.5ex},
          scaled y ticks = false,  
          scaled x ticks = false,   
		  legend pos=north west,
 		  samples=61,
    	  width=5in,
    	  height=3in,
          ymin=0,
          try min ticks =6,
          max space between ticks=15000pt,
          x tick label style=
          {
          /pgf/number format/fixed,
     	 /pgf/number format/1000 sep = \thinspace % Optional if you want to replace comma as the 1000 separator 
      	  },
      	  y tick label style=
          {
          /pgf/number format/fixed,
     	 /pgf/number format/1000 sep = \thinspace % Optional if you want to replace comma as the 1000 separator 
      	  },
      	  xmin=0
}

 \tableofcontents 
 \section{Repozytorium}
 Cały kod oraz sprawozdanie jest dostępne na repozytorium:
 \url{https://github.com/MenosGrandes/Doktoranckie}
\section{Informacje o sprzęcie testowym}
Do pomiaru czasu zostały wykorzystane funkcje \textit{QueryPerformanceCounter} oraz \textit{QueryPerformanceFrequency} z biblioteki \textbf{windows.h}. Wszystkie testy zostały wykonane na maszynie z systemem Windows 8.1Pro 64, z procesorem Intel(R) Core(TM) i7-5700HQ CPU @ 2.70GHz. Testy zostały przeprowadzone na kompilatorze \textit{mingw32-c++.exe (GCC) 5.3.0} z flagą \textit{-std=gnu++11} zapewniającą wsparcie \textit{C++11}.
\section{Quicksort}
Sortowanie to polega na wybraniu jednego elementu $p$, który może być środkowym, skrajnie lewym, skrajnie prawym lub losowym elementem zbioru. W następnym kroku elementy nie większe niż $p$ zostają ustawione na lewo tej wartości, a nie mniejsze na prawo. W ten sposób powstaną nam dwie części tablicy (niekoniecznie równe), gdzie w pierwszej części znajdują się elementy nie większe od drugiej. Następnie każdą z tych pod tablic sortujemy osobno według tego samego schematu. 
\subsection{Klasyfikacja}

\subsection{Oceny zlozonosci}
j
\subsubsection{Optymistyczna}
Jeżeli funkcja dzieląca zbiory o rozmiarze $n$ będzie dzieliła je zawsze na prawie równe części o rozmiarze $\frac{n}{2}$, można ustalić ze jest to najlepszy przypadek dla tego sortowania. Dzielenie takie następuje gdy wartość $p$ będzie mediana zbioru \cite{quicksort}
\begin{equation*}
T(n) = 2T(\frac{n}{2}) +\alpha n
\end{equation*}
Zakładając ze powyższa funkcja jest funkcja rekurencyjna następują zmiany dla kolejnej iteracji
\begin{align*}
 T(n) &= 2(2T(\frac{n}{4}) + \alpha \frac{n}{2}) + \alpha n \\
      &= 2^{2}T( \frac{n}{4}) + 2 \alpha n \\
      &= 2^{3}T( \frac{n}{8}) + 3 \alpha n
 \end{align*}
Rekurencyjne iteracje będą wykonywane do momentu $n = 2^{k}$, gdzie $\alpha$ jest  stalą, $k$ ilością iteracji, a $n$ rozmiarem zbioru początkowego. Czyli przyjmując ze $k = \log n$ :
\begin{equation*}
T(n) =nT(1) +\alpha n \log n
\end{equation*}
W notacji duze O bedzie to zatem:
\begin{equation*}
O(n \log n)
\end{equation*}
\subsubsection{Pesymistyczna}
Najgorszy przypadek w tym algorytmie występuje w sytuacji kiedy algorytm będzie dzielił tablice na pod tablice w sytuacji kiedy jedna z nich będzie zawierała 1 element.
\begin{equation*}
T(n) =T(1) +T(n−1) + \alpha n
\end{equation*}
Dla każdej iteracji rekurencji wyrażenie to będzie przedstawione jako 
\begin{align*}
T(n) &=T(n−1) +T(1) + \alpha n \\
	 &= T(n−2) +T(1) +\alpha(n−1)] +T(1) + \alpha n \\
	 &= T(n−3) + 3T(1) +\alpha(n−2 +n−1 +n) \\
	 &= T(n−i) +iT(1) +\alpha(n−i+ 1 + \cdot + n−2 +n−1 +n) \\
	 &= T(n−i) +iT(1) +\alpha \sum_{j=0}^{n-2}(n-j) \\ 
	 &= nT(1) +\alpha(n(n−2)−\frac{(n−2)(n−1)}{2})
 \end{align*}
Rekurencja wykonuje się dopóki $i=n-1$, ponieważ nie możne być sytuacji gdzie $n-i <0$.
Zlozonosc obliczeniowa wychodzi wiec :
\begin{equation*}
O(n^2)
\end{equation*}
\subsubsection{Średnia}
j

\section{Mergesort}
j
\subsection{Oceny zlozonosci}
j
\subsubsection{Optymistyczna}
j
\subsubsection{Pesymistyczna}
i
\subsubsection{Średnia}
h
\section{Heapsort}
j
\subsection{Oceny zlozonosci}
l
\subsubsection{Optymistyczna}
h
\subsubsection{Pesymistyczna}
o
\subsubsection{Średnia}
u
\section{ShellSort}
f
\subsection{Oceny zlozonosci}
ui
\subsubsection{Optymistyczna}
d'
\subsubsection{Pesymistyczna}
\subsubsection{Średnia}
\section{Bitonic sort}
\subsection{Oceny zlozonosci}
\subsubsection{Optymistyczna}
\subsubsection{Pesymistyczna}
\subsubsection{Średnia}
\section{Autorski algorytm}
\subsection{Oceny zlozonosci}
\subsubsection{Optymistyczna}
\subsubsection{Pesymistyczna}
\subsubsection{Średnia}

\begin{thebibliography}{99}
\bibitem{quicksort} \url{https://www.cise.ufl.edu/class/cot3100fa07/quicksort_analysis.pdf}
\end{thebibliography}
\end{document}
