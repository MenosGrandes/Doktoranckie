%%%%%%%%%%%%%%%%%%%%%%%%%%%%%%%%%%%%%%%%%
% University Assignment Title Page 
% LaTeX Template
% Version 1.0 (27/12/12)
%
% This template has been downloaded from:
% http://www.LaTeXTemplates.com
%
% Original author:
% WikiBooks (http://en.wikibooks.org/wiki/LaTeX/Title_Creation)
%
% License:
% CC BY-NC-SA 3.0 (http://creativecommons.org/licenses/by-nc-sa/3.0/)
% 
% Instructions for using this template:
% This title page is capable of being compiled as is. This is not useful for 
% including it in another document. To do this, you have two options: 
%
% 1) Copy/paste everything between \begin{document} and \end{document} 
% starting at \begin{titlepage} and paste this into another LaTeX file where you 
% want your title page.
% OR
% 2) Remove everything outside the \begin{titlepage} and \end{titlepage} and 
% move this file to the same directory as the LaTeX file you wish to add it to. 
% Then add \input{./title_page_1.tex} to your LaTeX file where you want your
% title page.
%
%%%%%%%%%%%%%%%%%%%%%%%%%%%%%%%%%%%%%%%%%
%\title{Title page with logo}
%----------------------------------------------------------------------------------------
%	PACKAGES AND OTHER DOCUMENT CONFIGURATIONS
%----------------------------------------------------------------------------------------

\documentclass[12pt]{article}
\usepackage{documentation}
\begin{document}

\begin{titlepage}

\newcommand{\HRule}{\rule{\linewidth}{0.5mm}} % Defines a new command for the horizontal lines, change thickness here

\center % Center everything on the page
 
%----------------------------------------------------------------------------------------
%	HEADING SECTIONS
%----------------------------------------------------------------------------------------

\textsc{\LARGE Politechnika Łódzka}\\[1.5cm] % Name of your university/college
\textsc{\Large Szybkie algorytmy}\\[0.5cm] % Major heading such as course name

%----------------------------------------------------------------------------------------
%	TITLE SECTION
%----------------------------------------------------------------------------------------

\HRule \\[0.4cm]
{ \huge \bfseries Badanie efektywności szybkich algorytmów sortowania  }\\[0.4cm] % Title of your document
\HRule \\[1.5cm]
 
%----------------------------------------------------------------------------------------
%	AUTHOR SECTION
%----------------------------------------------------------------------------------------



\begin{flushleft}\large

\begin{center} \emph{Prowadzący zajęcia:} \end{center}
\begin{center}
prof. dr hab. Mykhaylo \textsc{Yatsymirskyy}
\end{center} % Supervisor's Name
\end{flushleft}


% If you don't want a supervisor, uncomment the two lines below and remove the section above
\Large \emph{Autor:}\\
Filip \textsc{Rynkiewicz}\\[2cm] % Your name


%----------------------------------------------------------------------------------------
%	LOGO SECTION
%----------------------------------------------------------------------------------------

\includegraphics[scale=0.36]{logo.png}\\[1.3cm]
 

%----------------------------------------------------------------------------------------
%	DATE SECTION
%----------------------------------------------------------------------------------------

{\large \today}\\[5cm] % Date, change the \today to a set date if you want to be precise

\vfill % Fill the rest of the page with whitespace

\end{titlepage}






 \tableofcontents 
 \section{Repozytorium}
 Cały kod oraz sprawozdanie jest dostępne na repozytorium:
 \url{https://github.com/MenosGrandes/Doktoranckie}
\section{Informacje o sprzęcie testowym}
Do pomiaru czasu zostały wykorzystane funkcje \textit{QueryPerformanceCounter} oraz \textit{QueryPerformanceFrequency} z biblioteki \textbf{windows.h}. Wszystkie testy zostały wykonane na maszynie z systemem Windows 8.1Pro 64, z procesorem Intel(R) Core(TM) i7-5700HQ CPU @ 2.70GHz. Testy zostały przeprowadzone na kompilatorze \textit{mingw32-c++.exe (GCC) 5.3.0} z flagą \textit{-std=gnu++11} zapewniającą wsparcie \textit{C++11}.


\section{Quicksort}
Sortowanie to polega na wybraniu jednego elementu $p$, który może być środkowym, skrajnie lewym, skrajnie prawym lub losowym elementem zbioru. W następnym kroku elementy nie większe niż $p$ zostają ustawione na lewo tej wartości, a nie mniejsze na prawo. W ten sposób powstaną nam dwie części tablicy (niekoniecznie równe), gdzie w pierwszej części znajdują się elementy nie większe od drugiej. Następnie każdą z tych pod tablic sortujemy osobno według tego samego schematu. 
\subsection{Klasyfikacja}
\subsubsection{Optymistyczna}
Jeżeli funkcja dzieląca zbiory o rozmiarze $n$ będzie dzieliła je zawsze na prawie równe części o rozmiarze $\frac{n}{2}$, można ustalić ze jest to najlepszy przypadek dla tego sortowania. Dzielenie takie następuje gdy wartość $p$ będzie mediana zbioru \cite{quicksort}
\begin{equation*}
T(n) = 2T(\frac{n}{2}) +\alpha n
\end{equation*}
Zakładając ze powyższa funkcja jest funkcja rekurencyjna następują zmiany dla kolejnej iteracji
\begin{align*}
 T(n) &= 2(2T(\frac{n}{4}) + \alpha \frac{n}{2}) + \alpha n \\
      &= 2^{2}T( \frac{n}{4}) + 2 \alpha n \\
      &= 2^{3}T( \frac{n}{8}) + 3 \alpha n
 \end{align*}
Rekurencyjne iteracje będą wykonywane do momentu $n = 2^{k}$, gdzie $\alpha$ jest  stalą, $k$ ilością iteracji, a $n$ rozmiarem zbioru początkowego. Czyli przyjmując ze $k = \log n$ :
\begin{equation*}
T(n) =nT(1) +\alpha n \log n
\end{equation*}
W notacji duze O bedzie to zatem:
\begin{equation*}
O(n \log n)
\end{equation*}
\subsubsection{Pesymistyczna}
Najgorszy przypadek w tym algorytmie występuje w sytuacji kiedy algorytm będzie dzielił tablice na pod tablice w sytuacji kiedy jedna z nich będzie zawierała 1 element.
\begin{equation*}
T(n) =T(1) +T(n−1) + \alpha n
\end{equation*}
Dla każdej iteracji rekurencji wyrażenie to będzie przedstawione jako 
\begin{align*}
T(n) &=T(n−1) +T(1) + \alpha n \\
	 &= T(n−2) +T(1) +\alpha(n−1)] +T(1) + \alpha n \\
	 &= T(n−3) + 3T(1) +\alpha(n−2 +n−1 +n) \\
	 &= T(n−i) +iT(1) +\alpha(n−i+ 1 + \cdot + n−2 +n−1 +n) \\
	 &= T(n−i) +iT(1) +\alpha \sum_{j=0}^{n-2}(n-j) \\ 
	 &= nT(1) +\alpha(n(n−2)−\frac{(n−2)(n−1)}{2})
 \end{align*}
Rekurencja wykonuje się dopóki $i=n-1$, ponieważ nie możne być sytuacji gdzie $n-i <0$.
Zlozonosc obliczeniowa wychodzi wiec :
\begin{equation*}
O(n^2)
\end{equation*}
\subsubsection{Średnia}
Zakładając że piwot $p$ będzie wybierany tak aby partycjonowanie zbioru $V $ dzieliło go na zbiory $V_1$ i $V_2$ to zawsze prawdą będzie że długość tych podzbiorów będzie zawierała się w zbiorze od $0$ do $ \norm{V}-1$ \cite{quicksortAVG}.

Aby posortować \textit{n} elementów potrzeba \textit{T} czasu :
\begin{equation*}
T(n)=T(i)+T(n-i)+c \cdot n
\end{equation*}
Ponieważ $i$ jest wartośćią z przedziału $0$ do $n-1$, średnia wartość $T$ to :
\begin{equation*}
T(i)=\frac{1}{n}\sum_{j=0}^{n-1}T(j)
\end{equation*}
Skoro $n-i$ może mieć taką samą wartosć jak $i$:
\begin{equation*}
T(n-1) = T(i)
\end{equation*}
Zatem :
\begin{equation*}
T(n)=\frac{2}{n} \Bigg( \sum_{j=0}^{n-1}T(j)\Bigg) + c \cdot n
\end{equation*}
Usuwając $n$ z ułamka  :
\begin{equation*}
n \cdot T(n)=2 \cdot \Bigg( \sum_{j=0}^{n-1}T(j)\Bigg) + c \cdot n^{2}
\end{equation*}
Ponieważ równanie musi być prawdziwe dla każdej liczby trzeba zamienić $n$ na przykład na $n-1$ :
\begin{equation*}
(n-1) \cdot T(n-1)=2 \cdot \Bigg( \sum_{j=0}^{n-2}T(j)\Bigg) + c \cdot (n-1)^{2}
\end{equation*}
Wiec prawdą będzie :
\begin{align*}
n \cdot T(n)&=2 \cdot \Bigg( \sum_{j=0}^{n-1}T(j)\Bigg) + c \cdot n^{2}\\
(n-1) \cdot T(n-1)&=2 \cdot \Bigg( \sum_{j=0}^{n-2}T(j)\Bigg) + c \cdot (n-1)^{2}
 \end{align*}
 Odejmując drugie równanie od pierwszego, usuwając $c$ i grupując równanie otrzymamy :
 \begin{equation*}
 nT(n)=(n+1)T(n-1)+2cn
 \end{equation*}
Rozwiązując równanie rekurencyjne i grupując je otrzymamy :
\begin{equation*}
 \frac{T(n)}{(n+1)} = \frac{T(1)}{2} + 2 c \sum_{j=3}^{n+1} \frac{1}{j}
\end{equation*}
Ponieważ $n$ staję się bardzo duże, $\sum_{j=3}^{n+1} \frac{1}{j}$ dochodzi do $\ln(n) +\gamma $, gdzie $\gamma$ jest stałą Eulera.\\
Zatem :
\begin{equation*}
 \frac{T(n)}{n+1} = \frac{T(1)}{2} + 2 c \cdot \ln(n) + 2 c \gamma = \ln(n) + c_2 = O \ln(n) 
\end{equation*}
Złozoność wynosi zatem:
\begin{equation*}
 T(n)=O (n \cdot \ln(n)) 
 \end{equation*}

                % omit the '.tex' extension
\section{MergeSort}
Polega ona na podziałach sortowanego zbioru $V$ na równe części, i rekurencyjnych dalszych podziałów  do momentu w którym nie będzie można dzielić dalej podtablic, czyli do momentu w którym dzielona tablica będzie miała rozmiar większy niż 3.
\subsection{Klasyfikacja}
\subsubsection{Optymistyczna}
Najlepszy przypadek następuje gdy tablica która algorytm ma posortować jest już posortowana.
\subsubsection{Pesymistyczna}

\subsubsection{Średnia}
                % omit the '.tex' extension
\section{HeapSort}
Sortowanie przez kopcowanie polega na wykorzystaniu kopca.\\
 Podstawowymi akcjami wykonywanymi przez algorytm jest:
\begin{itemize}
\item Tworzenie kopca,
\item sortowanie.
\end{itemize}

\subsubsection{Kopiec}
Jest to kontener danych podobny do drzewa binarnego\cite{heapsort}. Każdy element w kopcu posiada 3 inne elementu : swojego przodka oraz dwójkę dzieci, element \textit{root} nie posiada przodka ponieważ jest pierwszym elementem. 
Element \textit{przodek} zawsze posiada wartość większa od wartości swoich dzieci.	\\

Aby zawsze na pierwszym miejscu znajdował się maksymalny element zbioru kopiec posiada metodę o nazwie $Heapify$. Jest ona odpowiedzialna za stworzenie takiego kopca którego każdy element jest uporządkowany tak aby wartość rodzica była wartością większa niż wartości jego dzieci.\\
Złożoność obliczeniowa takiej funkcji to :
\begin{align*}
T(N) &= T(\frac{2n}{3}) + O(1) \\
T(N) &=O(\ln n)
\end{align*} 
Lecz dla każdego elementu na wysokości $h$ złożoność tej funkcji będzie wynosić $O(h)$
\par Kolejna funkcja wykorzystywana w tym sortowaniu jest tworzenie kopca $BuildHeap$.
W metodzie tej dla każdego elementu wykonywana jest metoda $Heapify$ która porządkuje kopiec. \\
\par Dla $n$ elementowego kopca, którego wysokość $h$ to $\floor*{\ln n}$ w elemencie $\ceil*{\frac{n}{2^{h+1}}}$ złożoność obliczeniowa funkcji $BuildHeap$ będzie wynosić 
\begin{equation*}
\sum_{h=0}^{\floor*{\ln n}}\ceil*{\frac{n}{2^{h+1}}}O(h) = O \Bigg( n \sum_{h=0}^{\floor*{\ln n}}\ceil*{\frac{h}{2^{h}}} \Bigg)
\end{equation*}
Rozwijając powyższe równanie dla nieskończenie wielu elementów możemy otrzymać:
\begin{equation*}
\sum_{h=0}^{\infty}\frac{h}{2^h} = \frac{\frac{1}{2}} {(1-\frac{1}{2})^2} = 2
\end{equation*}
Oznacza to że możemy zbudować maksymalny kopiec w czasie liniowym $O(n)$.\\
Biorąc pod uwagę wszystkie powyższe wnioski samo sortowanie przez kopcowanie będzie miało żlożoność obliczeniową:
\begin{equation*}
O(n \ln n)
\end{equation*}
Ponieważ procedura budowania maksymalnego kopca zajmuję $O(n)$, a dla elementów $n-1$ trzeba wywołać $Heapify$ która zajmuję $O(\ln n)$.

                % omit the '.tex' extension


\section{Shellsort}
Jest to algorytm sortowania stanowiący uogólnienie sortowania przez wstawianie\cite{shell}. Dopuszcza on sortowanie elementów znajdujących się od siebie o określona odległość. W tej implementacji algorytmu został wykorzystany ciąg Marcina Ciury :
\begin{equation*}
 	1 , 4 , 10 , 23 , 57 , 132 , 301 , 701 
\end{equation*}
\subsection{Klasyfikacja}
\subsubsection{Optymistyczna}
Ten przypadek występuje gdy zbiór który ma być posortowany jest już posortowany.\\
Zakładając ze posortowanie elementów oddzielonych od siebie o jedna przerwę potrzeba $n$ porównań. Dla elementów oddalonych od siebie o 3 potrzeba $3 \cdot \frac{n}{3}$ ,
dla elementów oddalonych od siebie o 7 potrzeba $7 \cdot \frac{n}{7}$ . Ogólny wzór będzie zatem taki:
\begin{equation*}
2^k -1 < n
\end{equation*}
tak wiec 
\begin{equation*}
k < \ln(n+1)
\end{equation*}
Powyższe wzory oznaczają ze w najlepszym przypadku optymistyczna złożoność nie będzie mniejsza niż 
\begin{equation*}
n \cdot \ln(n+1) = O(n\cdot \ln(n))
\end{equation*}
\subsubsection{Pesymistyczna}
Ponieważ najgorszy przypadek zawsze zależny od ciągu przerw przyjętego przy algorytmie nie można określić jednego wzoru należny określić jeden ogólny.
Zakładając ze w najgorszym przypadku potrzeba $n^2$ porównań dla elementów oddalonych od siebie o jedna przerwę, dla elementów oddzielonych o 3 potrzeba $3 \cdot (\frac{n}{3})^2$.
Ogólny wzór można wysnuć poprzez porównywanie ze sobą dwóch ciągów
\begin{align*}
& n^2 \cdot (1 +\frac{1}{3}+\frac{1}{7}+\frac{1}{15}+\frac{1}{31} \ldots )\\
&< n^2 \cdot (1 +\frac{1}{2}+\frac{1}{4}+\frac{1}{8}+\frac{1}{16} \ldots )\\
& = n^2 \cdot  2
\end{align*}
\subsubsection{Średnia}
Ilość porównań dla średniego przypadku zależny od ciągu przyjętego przez algorytm. Ciąg Ciury\cite{shellCiura} nie ma ogólnego wzoru przyjętego dla tego przypadku. Przyjmuje się ze algorytm ten osiąga najlepsza złożoność średnia $O(n \cdot \ln(n))$.


                % omit the '.tex' extension


\section{BitonicSort}
\begin{equation}
\end{equation}
\subsection{Klasyfikacja}
\subsubsection{Optymistyczna}

\subsubsection{Pesymistyczna}

\subsubsection{Średnia}


                % omit the '.tex' extension
\section{Optymalizacja QuickSort}
QuickSort jest algorytmem implementowanym w postaci rekurencyjnej. Minusem tego rozwiązania jest tworzenie nowego stosu dla każdej instancji funkcji co spowalnia algorytm i zajmuje pamięć. Usuwając wywołania rekurencyjne algorytm przyspiesza oraz zmniejsza sie ilosc pamieci wykorzystana do posortowania zbioru.                % omit the '.tex' extension


\begin{thebibliography}{99}
\bibitem{quicksort} \url{https://www.cise.ufl.edu/class/cot3100fa07/quicksort_analysis.pdf}
\bibitem{quicksortAVG}\url{https://secweb.cs.odu.edu/~zeil/cs361/web/website/Lectures/quick/pages/ar01s05.html}
\bibitem{mergesort}\url{http://cs.fit.edu/~pkc/classes/writing/hw13/luis.pdf}
\bibitem{mergesortAVG}\url{https://www.cs.duke.edu/courses/fall14/cps130/notes/scribe2.pdf}
\end{thebibliography}
\end{document}
