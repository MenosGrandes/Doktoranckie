\section{MergeSort}
Polega ona na podziałach sortowanego zbioru $V$ na równe części, i rekurencyjnych dalszych podziałów  do momentu w którym nie będzie można dzielić dalej pod tablic, czyli do momentu w którym dzielona tablica będzie miała rozmiar większy niż 3 \cite{mergesort}.
\subsection{Klasyfikacja}
\subsubsection{Optymistyczna}
Najlepszy przypadek następuje gdy zbiór który algorytm ma posortować jest już posortowany.
Ponieważ tylko jeden element z list jest porównywany, ilość porównań zmniejsza się o $\frac{N}{2}$.

\begin{align*}
T(N) &= 2T(\frac{N}{2}) +\frac{N}{2}  \\
 &= 2\Big(2T(\frac{N}{4})+ \frac{N}{4} \Big) + \frac{N}{2} \\
 &= 4\Big(2T(\frac{N}{8})+ \frac{N}{8} \Big) + N \\
 &= 8T(\frac{N}{8})+ \frac{3N}{2}
\end{align*}
Co daje przy przeliczeniu na ogólny przypadek :
\begin{align*}
T(N) &= 2^k T\Big(\frac{N}{2^k} \Big) + \frac{kN}{2} \\
&=\frac{N}{2}log_2 N
\end{align*}
Z powyższych równań wynika ze złożoność obliczeniowa dla najlepszego przypadku wynosi :
\begin{equation*}
O(N \cdot \log(N))
\end{equation*}
\subsubsection{Pesymistyczna}
Sytuacja w której podczas każdego kroku dzielenia zbiory podzielone są tak ze dwie największe wartości znajdują się w oddzielnych pod tablicach
Ponieważ jest to algorytm, przeważnie, rekurencyjny jego złożoność można przedstawić za pomocą takich wzorów:
\begin{align*}
T(N) &= 2T(\frac{N}{2}) + N -1 \\
 &= 2\Big(2T(\frac{N}{4})+ \frac{N}{2} -1 \Big) + N -1 \\
 &= 4\Big(2T(\frac{N}{8})+ \frac{N}{4} -1 \Big) + 2N -3 \\
 &= 8T(\frac{N}{8})+ N + + N+ N -4 - 2 - 1 
\end{align*}
$N-1$ oznacza sytuacje w której unika się tylko jednego porównania.\\

Z powyższego wyprowadzenia można wysnuć wniosek ze :
\begin{equation*}
T(N) = 2^k T\Big(\frac{N}{2^k} \Big) +kN -(2^k -1)
\end{equation*}
gdzie $k$ jest ilością wykonań rekurencji.\\
\par Zakładając ze $T(1) = 0$ wynika ze :
\begin{align*}
2^k &= N \\
k &= \log_2 N \\
T(N) &= N log_2 N -N -1
\end{align*}
Z powyższych równań wynika ze złożoność obliczeniowa dla najgorszego przypadku wynosi :
\begin{equation*}
O(N \cdot \log(N))
\end{equation*}

\subsubsection{Średnia}
Poniewaz w tym algorytmie nie ma wybierania zadnych losowych elementow po ktorych dzieli sie podtablice, zlozonosc obliczeniowa dla przypadku sredniego jest taka sama jak dla przypadku najgorszego\cite{mergesortAVG}.